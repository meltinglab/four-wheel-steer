% !TEX encoding = UTF-8 Unicode
% !TEX TS-program = pdflatex
% !TEX spellcheck = en-US
% !TEX root = ../Report.tex

\chapter{Introduction}
	Nowadays, many urban car manufactures are implementing the four-wheel steering system control on board. In this new vehicle generation, not only the front wheels are steerable but also the rear ones can be steered too. As a consequence, the system will be characterized by a higher number of controlling inputs under equal number of states. This situation increases the maneuverability of the vehicle itself and provides the possibility of optimization or constraint satisfaction during the vehicle transfer between two definite boundaries.
	
	In this paper, we are going to present the kinematic and dynamic modeling of a four-wheel steering vehicle. Moreover, we will show both the step-by-step procedure we adopted to design an optimal controller, by means of the Linear Quadratic Regulator (LQR), and the software tools we employed to achieve a correct result by way of team work. The objective is to create a portable control, working independently from other control systems, that could be integrated in any vehicle. It takes as input the state of the vehicle and, by acting on the steering angle of the rear wheels, it will drive the car in different working conditions. Therefore, the control system will always try to fit some reference points, moment by moment.The correctness of modeling and the designed optimal controller efficiency are going to be verified by the aid of a specific pre-defined vehicle simulator we chose.
	
	Within the following chapters, we will describe also how we managed some critical situations, the tools we adopted, the problems we dealt with and the solutions we have implemented according to solve them. 
	
	Finally, we will discuss the reached results. We will firstly exhibit the achieved outcomes in terms of system compliance. There will be both a brief description of the coding standards we dealt with and how we managed the automatic code and documentation generation tools. Than, we are going to display the main differences, in terms of performance, between the Open and Closed-loop operating conditions. Afterwards, there follow our final considerations about them. 
